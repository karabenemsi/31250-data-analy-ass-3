% !TeX spellcheck = en_US
% !TEX root = Ausarbeitung.tex

\section{The data mining problem}

In the given scenario we have received a data set. This contains data points for various customers of an insurance company. The aim of data mining is to predict whether a customer will purchase an insurance. 

The dataset contains multiple features, that have been explored in Assesment 2 of this subject. The list of features contains different type of features. Some of them are dichotomous, some are categorical and some are rational or ordinal features. This wide variety will require some preprocessing to build a good model. Most of the features contain characters as values, one of them contains a date in australian format.

The goal for this problem will be to categorize each datapoint. There are only two categories available: 1 for "will purchase insurance" and 0 for "will not purchase insurance".


\section{Preprocessing and transformation}

To provide a good starting point for the classifiers I preprocess the raw data. To solve the problem I use python with jupyter notebooks. The following explanations will be supported by short listings. The main sourcecode can be found in the appendix of this report.

\subsection{Parsing the data}
After importing the data from csv, I start by converting the date into a unix timestamp and all ordinal features into numbers. I also convert dichotomous features that are coded with "Y" and "N" into machine readable "0" and "1", respectively.

\lstinputlisting[language=Python,  linerange={41-43,54-57}, caption={An excerpt of the parse function}]{../helpers.py}

\subsection{Categorical data}
After parsing all data I convert categorical features into many flags. This makes it easier for following classifiers to work with these features. To do this I use a function of \texttt{pandas} called \texttt{get\_dummies}. After adding those flags I delete the original feature as it would contain redundant information. The shown function can handle multiple feature conversions at once.

\lstinputlisting[language=Python,  linerange={60-72}, caption={The function to convert one categorical feature into many dichotomous}]{../helpers.py}

After converting the features, the training set and and test set could contain different amount of features (flags). To solve this, I populate the sets with all missing flags.

\lstinputlisting[language=Python,  linerange={29-33}, caption={Add missing flag features to both sets}]{../preprocess.py}

\subsection{Feature Selection}
The next preprocessing step is feature selection. I remove different features depending on different reasons. The first removed feature is \texttt{Personal\_info5} which contains lots of empty values and a low variance. After doing so I delete all rows with empty values inside the training set and fill all empty values inside the test set. I delete emtpy values inside the training set and fill them inside the test set because the empty values inside the training set are more numerous and appear in different features. Inside the test set they only appear in two dichotomous features and less often.

\lstinputlisting[language=Python, linerange={37-43}, caption={Removal of specifc features and emtpy values}]{../preprocess.py}

Next I remove all features with a variance under $0.16$. I remove those because they don't contain enough information to classify the data and only increase the dimensionality of the data set. The feature I remove in the training set, I also remove in the test set. 

\lstinputlisting[language=Python, linerange={75-89}, caption={Removal of features with low variance}]{../helpers.py}

\subsection{Outliers}

To remove all outliers inside the training set, I calculate the z-score for most features and delete all rows on the edge of the normal distribution (z-score > 3).
I don't alter the test set in this step

\subsection{Scaling}

To generate a scaled and normalized data set for the classifers to work with, I use the \texttt{StandardScaler} of scikit-learn. This will scale the features to unit variance and center them to have a mean of zero. The output of the Standard Scaler will be more gaussian than the input data and provide a better starting point for the classifiers to work with. Some of them will also require scaled data.

I tried out different scaling methods but the combination of these two got the best results for me. This makes sense as the resulting values will be distributed normaly and most classifiers work best with this kind of data.


\section{Classification techniques}

For the classification process I reviewed multiple classifiers. I ended up optimizing three of them.

After reading and preprocessing the data set, I am splitting the training set int a train and test part. The test part in this split takes up 30\,\%.

\lstinputlisting[language=Python, linerange={59-61}, caption={Preprocessing and spliting the dataset}]{../main.py}

For each classifier I use the SMOTE technique to oversample our dataset. SMOTE can be used on unbalanced datasets as ours (we less buyers than other). SMOTE is a method to oversample a data set to improve ROC performance \cite{SMOTE}.

\lstinputlisting[language=Python, linerange={12-19}, caption={Function with SMOTE to oversample the dataset}]{../main.py}


\subsection{Parameter optimization}

To improve the performance of all classifiers I use \texttt{GridSearchCV} to find the optimal parameters for each classifier. The parameters shown in the following sections are all found using this technique. \texttt{GridSearchCV} runs the classifier with every possible combination of parameters and can return the parameter with the highest accuracy.

\begin{lstlisting}[language=Python, caption={Example use of \texttt{GridSearchCV} for the random forest classifier}]
model = RandomForestClassifier(n_jobs=-1)
params = {'n_estimators':range(0,200), 'criterion':('gini','entropy')}
gridSearch = GridSearchCV(model, params, cv=5, verbose=2, n_jobs=-1)
gridSearch.fit(X_g_train, y_g_train)
gridSearch.cv_results_['params'][gridSearch.best_index_]
\end{lstlisting}

\subsection{Random Forest}

The random forest is an ensemble classification method. It will build many decision trees and train them with a subset of the supplied data. After running those trees it will collect votes from all trees to classify a data point.

The selected parameters set the number of trees inside the forest to $185$. This number gave optimal results without overfitting. Overfitting with too many trees can have an impact on the performance of the random forest. The selected criterion "entropy" uses information gain to build the decision trees.

\lstinputlisting[language=Python, linerange={71-74}, caption={\texttt{RandomForestClassifier} used for classifing}]{../main.py}

\subsection{Support Vector Machines}

A SVM tries to find a hyperplane in a n-dimensional room, where n is the number of features provided. The hyperplane should distinctly classify all data points. It will look for the hyperplane with the biggest distance between points of each class.

\lstinputlisting[language=Python, linerange={90-93}, caption={\texttt{SVC} used for classifing}]{../main.py}

\subsection{Multilayer Perceptron}

Neural Networks with multiple fully connected single neurons are called Multilayer Perceptron Networks (MLP networks). They are feed forward artificial neural networks and uses backpropagation to learn in each iteration of its training. The selected network for this data mining problem consists of 4 layers. The two hidden ones are specified inside the parameters of the constructor.

\lstinputlisting[language=Python, linerange={101-105}, caption={\texttt{MLPClassifier} used for classifing}]{../main.py}

\section{The best classifier}

For the given data set and problem the random forest achieves the best results for me. I therefore selected the random forest as the best classifier. The problem is a unbalanced data set and is not easily linearly solvable. Therefore the Vector machine has its problems with this type of data set. The MLP is not able to reach the performance of the random forest in my case.

\subsection{Classifer type}
As described above, the random forest belongs to the ensemble classifiers. It obtains its result from a large number of decision trees and counts the votes together to make the classification. This increases the accuracy in comparison to a single tree and also overfitting becomes a minor problem. It is also true that with increased variation the accuracy increases to a certain degree.

\subsection{Performance}

Using the training set, the random forest reached a AUC of $0.76$. That's just a little bit better than the other algorithms. Increasing the number of trees did improve the performance on the training set but decreased the performance of the test set on kaggle. Using python to run the random forest the execution was stable and fast compared to SVM or the MLP. Using multiple runs of the random forest, I was able to improve the score up to $0.77$, this corresponds with the score on kaggle.

\subsection{Solving the problem}

Each decision tree inside the random forest uses multiple splitings to split the data set into different classes. Each split uses one or multiple features to divide the data set. Each split is optimized to get the optimal information gain for it. Inside the random forest each tree is trained with its own subset of the training data.

My final solution uses multiple execution of the random forest. After all executions a function collects all votes and turns them into a single class. The parameters for the random forest are optimized using grid search.

\lstinputlisting[language=Python, linerange={69-86}, caption={Final problem solving code}]{../main.py}







